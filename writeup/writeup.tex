\documentclass[a4paper]{scrartcl}

\usepackage[usenames,dvipsnames]{xcolor}

\title{Camera Based 2D Feature Tracking ReadMe}
\author{Philipp Rapp}
\date{\today}

\begin{document}

\maketitle

\section*{Mid-Term Report}
\subsection*{MP.0 Mid-Term Report}
\textcolor{gray}{\textit{Provide a Writeup / README that includes all the rubric points and how you addressed each one. You can submit your writeup as markdown or pdf.}}

The document at hand represents the readme file.

\section*{Data Buffer}
\subsection*{MP.1 Data Buffer Optimization}
\textit{Implement a vector for dataBuffer objects whose size does not exceed a limit (e.g. 2 elements). This can be achieved by pushing in new elements on one end and removing elements on the other end.}

In order to solve this task, I decided to implement a ring buffer.
A ring buffer can be implemented by creating a plain array with a fixed capacity (in this case 2 elements)
and keeping track of
\begin{itemize}
	\item the current first entry (the head) and
	\item the occupied slots (the size).
\end{itemize}
In order to have reusable code, I used a template for both the data type as well as
for the capacity.
In order to have the same ease of use as for the STL vector, I also implemented a (non-complete)
iterator class for the ring buffer.

\section*{Keypoints}
\subsection*{MP.2 Keypoint Detection}
\subsection*{MP.3 Keypoint Removal}

\section*{Descriptors}
\subsection*{MP.4 Keypoint Descriptors}
\subsection*{MP.5 Descriptor Matching}
\subsection*{MP.6 Descriptor Distance Ratio}

\section*{Performance}
\subsection*{MP.7 Performance Evaluation 1}
\subsection*{MP.8 Performance Evaluation 2}
\subsection*{MP.9 Performance Evaluation 3}


\end{document}